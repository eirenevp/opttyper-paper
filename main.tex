%% For double-blind review submission, w/o CCS and ACM Reference (max submission space)
\documentclass[sigplan,10pt,review,anonymous]{acmart}
\settopmatter{printfolios=true,printccs=false,printacmref=false}
%% For double-blind review submission, w/ CCS and ACM Reference
%\documentclass[sigplan,review,anonymous]{acmart}\settopmatter{printfolios=true}
%% For single-blind review submission, w/o CCS and ACM Reference (max submission space)
%\documentclass[sigplan,review]{acmart}\settopmatter{printfolios=true,printccs=false,printacmref=false}
%% For single-blind review submission, w/ CCS and ACM Reference
%\documentclass[sigplan,review]{acmart}\settopmatter{printfolios=true}
%% For final camera-ready submission, w/ required CCS and ACM Reference
%\documentclass[sigplan]{acmart}\settopmatter{}


%% Conference information
%% Supplied to authors by publisher for camera-ready submission;
%% use defaults for review submission.
\acmConference[PL'18]{ACM SIGPLAN Conference on Programming Languages}{January 01--03, 2018}{New York, NY, USA}
\acmYear{2018}
\acmISBN{} % \acmISBN{978-x-xxxx-xxxx-x/YY/MM}
\acmDOI{} % \acmDOI{10.1145/nnnnnnn.nnnnnnn}
\startPage{1}

%% Copyright information
%% Supplied to authors (based on authors' rights management selection;
%% see authors.acm.org) by publisher for camera-ready submission;
%% use 'none' for review submission.
\setcopyright{none}
%\setcopyright{acmcopyright}
%\setcopyright{acmlicensed}
%\setcopyright{rightsretained}
%\copyrightyear{2018}           %% If different from \acmYear

%% Bibliography style
\bibliographystyle{ACM-Reference-Format}
%% Citation style
%\citestyle{acmauthoryear}  %% For author/year citations
\citestyle{acmnumeric}     %% For numeric citations
%\setcitestyle{nosort}      %% With 'acmnumeric', to disable automatic
                            %% sorting of references within a single citation;
                            %% e.g., \cite{Smith99,Carpenter05,Baker12}
                            %% rendered as [14,5,2] rather than [2,5,14].
%\setcitesyle{nocompress}   %% With 'acmnumeric', to disable automatic
                            %% compression of sequential references within a
                            %% single citation;
                            %% e.g., \cite{Baker12,Baker14,Baker16}
                            %% rendered as [2,3,4] rather than [2-4].


%%%%%%%%%%%%%%%%%%%%%%%%%%%%%%%%%%%%%%%%%%%%%%%%%%%%%%%%%%%%%%%%%%%%%%
%% Note: Authors migrating a paper from traditional SIGPLAN
%% proceedings format to PACMPL format must update the
%% '\documentclass' and topmatter commands above; see
%% 'acmart-pacmpl-template.tex'.
%%%%%%%%%%%%%%%%%%%%%%%%%%%%%%%%%%%%%%%%%%%%%%%%%%%%%%%%%%%%%%%%%%%%%%


%% Some recommended packages.
\usepackage{booktabs}   %% For formal tables:
                        %% http://ctan.org/pkg/booktabs
\usepackage{subcaption} %% For complex figures with subfigures/subcaptions
                        %% http://ctan.org/pkg/subcaption
                        
                        
%%%%%%%%%%%%%%%%%%%%%%%%%%%%%%%%%%%%%%%%%%%%%%%%%%%%%%%%%%%%%%%%%%%%%%
\usepackage{graphpap,amscd,mathrsfs,graphicx,lscape,dsfont,bm,url,color}
\usepackage{verbatim}
\usepackage{parcolumns}
\usepackage{mathtools}
\usepackage[capitalise,nameinlink]{cleveref}
\usepackage{bold-extra}
\usepackage{threeparttable}
\usepackage{tabularx}
\usepackage{adjustbox}
\usepackage{caption}
\usepackage{subcaption}
%\usepackage{booktabs}
\usepackage{listings}
\usepackage{xspace}
\usepackage[inline]{enumitem}
\usepackage{bm}

\newcommand{\qqpi}[2]{[\![#2]\!]_{#1}}
\newcommand{\prodts}{\textsc{ProdTS}\xspace}
\newcommand\numberthis{\addtocounter{equation}{1}\tag{\theequation}}


%% JavaScript
% JavaScript support
\usepackage{listings}
\lstset{ %
backgroundcolor=\color{white}, % choose the background color; you must add \usepackage{color} or \usepackage{xcolor}
basicstyle=\normal, % the size of the fonts that are used for the code
breakatwhitespace=true, % sets if automatic breaks should only happen at whitespace
breaklines=true, % sets automatic line breaking
captionpos=b, % sets the caption-position to bottom
commentstyle=\color{mygreen}, % comment style
deletekeywords={...}, % if you want to delete keywords from the given language
%escapeinside={\%}, % if you want to add LaTeX within your code
escapeinside={*@}{@*}, % if you want to add LaTeX within your code
extendedchars=true, % lets you use non-ASCII characters; for 8-bits encodings only, does not work with UTF-8
%frame=single, % adds a frame around the code
keepspaces=true, % keeps spaces in text, useful for keeping indentation of code (possibly needs columns=flexible)
columns=flexible,
keywordstyle=\color{blue}, % keyword style
morekeywords={*,...}, % if you want to add more keywords to the set
numbers=left, % where to put the line-numbers; possible values are (none, left, right)
numbersep=2pt, % how far the line-numbers are from the code
numberstyle=\tiny\color{mygray}, % the style that is used for the line-numbers
rulecolor=\color{black}, % if not set, the frame-color may be changed on line-breaks within not-black text (e.g. comments (green here))
showspaces=false, % show spaces everywhere adding particular underscores; it overrides 'showstringspaces'
showstringspaces=false, % underline spaces within strings only
showtabs=false, % show tabs within strings adding particular underscores
stepnumber=1, % the step between two line-numbers. If it's 1, each line will be numbered
%stringstyle=\color{mymauve}, % string literal style
tabsize=2, % sets default tabsize to 2 spaces
title=\lstname % show the filename of files included with \lstinputlisting; also try caption instead of title
}

\lstdefinelanguage{JavaScript}{
keywords={const, typeof, new, true, false, catch, function, 
  return, null, catch, switch, var, if, in, while, do, else, 
  case, break, class, export,throw, implements, import, this,
  exports, interface, readonly},
keywordstyle=\color{myrose},
ndkeywords={boolean, string, number, any, Array, Array<any>},
ndkeywordstyle=\color{Maroon}\bfseries,
identifierstyle=\color{RoyalBlue},
sensitive=false,
%escapechar=!,
comment=[l]{//},
morecomment=[s]{/*}{*/},
commentstyle=\color{mygray}\ttfamily,
%stringstyle=\color{red}\ttfamily,
morestring=[b]',
morestring=[b]"
}
 
\newlength{\listingindent}                %declare a new length
\setlength{\listingindent}{\parindent} 

\lstset{
language=JavaScript,
extendedchars=true,
basicstyle=\scriptsize\ttfamily,
showstringspaces=false,
showspaces=false,
numbers=left,
numberstyle=\color{mygray}\tiny,
numbersep=1pt,
tabsize=2,
breaklines=true,
showtabs=false,
captionpos=b,
xleftmargin=\listingindent,         
framexleftmargin=\listingindent,    
framextopmargin=6pt,
framexbottommargin=6pt, 
frame=tlrb, framerule=0pt,
linewidth=\linewidth
}

\graphicspath{ {figs/} }

\begin{document}

%% Title information
\title[Probabilistic Type Inference]{Probabilistic Type Inference by Optimizing Logical and Natural Constraints}
%\title[Short Title]{Full Title}         %% [Short Title] is optional;
                                        %% when present, will be used in
                                        %% header instead of Full Title.
%\titlenote{with title note}             %% \titlenote is optional;
                                        %% can be repeated if necessary;
                                        %% contents suppressed with 'anonymous'
%\subtitle{Subtitle}                     %% \subtitle is optional
%\subtitlenote{with subtitle note}       %% \subtitlenote is optional;
                                        %% can be repeated if necessary;
                                        %% contents suppressed with 'anonymous'


%% Author information
%% Contents and number of authors suppressed with 'anonymous'.
%% Each author should be introduced by \author, followed by
%% \authornote (optional), \orcid (optional), \affiliation, and
%% \email.
%% An author may have multiple affiliations and/or emails; repeat the
%% appropriate command.
%% Many elements are not rendered, but should be provided for metadata
%% extraction tools.

%% Author with single affiliation.
\author{First1 Last1}
\authornote{with author1 note}          %% \authornote is optional;
                                        %% can be repeated if necessary
\orcid{nnnn-nnnn-nnnn-nnnn}             %% \orcid is optional
\affiliation{
  \position{Position1}
  \department{Department1}              %% \department is recommended
  \institution{Institution1}            %% \institution is required
  \streetaddress{Street1 Address1}
  \city{City1}
  \state{State1}
  \postcode{Post-Code1}
  \country{Country1}                    %% \country is recommended
}
\email{first1.last1@inst1.edu}          %% \email is recommended

%% Author with two affiliations and emails.
\author{First2 Last2}
\authornote{with author2 note}          %% \authornote is optional;
                                        %% can be repeated if necessary
\orcid{nnnn-nnnn-nnnn-nnnn}             %% \orcid is optional
\affiliation{
  \position{Position2a}
  \department{Department2a}             %% \department is recommended
  \institution{Institution2a}           %% \institution is required
  \streetaddress{Street2a Address2a}
  \city{City2a}
  \state{State2a}
  \postcode{Post-Code2a}
  \country{Country2a}                   %% \country is recommended
}
\email{first2.last2@inst2a.com}         %% \email is recommended
\affiliation{
  \position{Position2b}
  \department{Department2b}             %% \department is recommended
  \institution{Institution2b}           %% \institution is required
  \streetaddress{Street3b Address2b}
  \city{City2b}
  \state{State2b}
  \postcode{Post-Code2b}
  \country{Country2b}                   %% \country is recommended
}
\email{first2.last2@inst2b.org}         %% \email is recommended


%% Abstract
%% Note: \begin{abstract}...\end{abstract} environment must come
%% before \maketitle command
\begin{abstract} \label{sec:abstract}
  We present a new approach to the type inference problem for dynamic languages.
Our goal is to combine
logical constraints, that is, deterministic information from a type system,
with natural constraints, uncertain information about types from
sources like identifier names.
  %
  To this end, we introduce a framework for probabilistic type inference that combines logic and learning:
logical constraints on the types are extracted from the program, and deep learning is applied to predict types
from surface-level code properties that are statistically associated, such as variable names.
The main insight of our method is to constrain the predictions from the learning procedure 
to respect the logical constraints, which we achieve by relaxing the logical inference problem
of type prediction into a continuous optimization problem.
%   We combine logical constraints generated by a static analysis of the source code with natural constraints learned from
%   existing codebases into a single optimization problem.
%   %
%   % To do so, the key idea is to use a continuous representation of the logical constraints part that can be jointly optimized with the learned natural constraints.
%   The main insight of our method is to relax the problem of type inference into a problem of numerical continuous optimization.
  %problem by using a continuous representation of the logical constraints part.
  %
  To evaluate the idea, we build a tool called \prodts to predict a TypeScript declaration file for a JavaScript library.
  %
\prodts combines a continuous interpretation of logical constraints derived by a simple
  augmented static analysis of the JavaScript code, with natural constraints obtained from a deep learning
 model, which learns naming conventions for types from a large code base.
  %
  We evaluate \prodts on a data set of 5,800 open source JavaScript projects 
which have type annotations in 
  the well-known DefinitelyTyped repository. We find that combining logical
  and natural constraints yields a large improvement in performance
over either kind of information individually, and produces $50\%$ fewer incorrect
type predictions
than previous approaches from the research literature.
  %
  % By transcribing a type inference procedure into a numerical optimization problem we initiate a novice way to balance between hard and natural constraints for suggesting types and therefore contribute towards situations where developers efficiently achieve the best of both the dynamically and statically-typed world.
\end{abstract}


%% 2012 ACM Computing Classification System (CSS) concepts
%% Generate at 'http://dl.acm.org/ccs/ccs.cfm'.
\begin{CCSXML}
<ccs2012>
<concept>
<concept_id>10011007.10011006.10011008</concept_id>
<concept_desc>Software and its engineering~General programming languages</concept_desc>
<concept_significance>500</concept_significance>
</concept>
<concept>
<concept_id>10003456.10003457.10003521.10003525</concept_id>
<concept_desc>Social and professional topics~History of programming languages</concept_desc>
<concept_significance>300</concept_significance>
</concept>
</ccs2012>
\end{CCSXML}

\ccsdesc[500]{Software and its engineering~General programming languages}
\ccsdesc[300]{Social and professional topics~History of programming languages}
%% End of generated code


%% Keywords
%% comma separated list
\keywords{keyword1, keyword2, keyword3}  %% \keywords are mandatory in final camera-ready submission


%% \maketitle
%% Note: \maketitle command must come after title commands, author
%% commands, abstract environment, Computing Classification System
%% environment and commands, and keywords command.
\maketitle


\section{Introduction}

Text of paper \ldots


%% Acknowledgments
\begin{acks}                            %% acks environment is optional
                                        %% contents suppressed with 'anonymous'
  %% Commands \grantsponsor{<sponsorID>}{<name>}{<url>} and
  %% \grantnum[<url>]{<sponsorID>}{<number>} should be used to
  %% acknowledge financial support and will be used by metadata
  %% extraction tools.
  This material is based upon work supported by the
  \grantsponsor{GS100000001}{National Science
    Foundation}{http://dx.doi.org/10.13039/100000001} under Grant
  No.~\grantnum{GS100000001}{nnnnnnn} and Grant
  No.~\grantnum{GS100000001}{mmmmmmm}.  Any opinions, findings, and
  conclusions or recommendations expressed in this material are those
  of the author and do not necessarily reflect the views of the
  National Science Foundation.
\end{acks}


%% Bibliography
%\bibliography{bibfile}


%% Appendix
\appendix
\section{Appendix}

Text of appendix \ldots

\end{document}
